\documentclass[aspectratio=43,14pt,spanish]{beamer}

\usepackage[utf8]{inputenc}
\usepackage[T1]{fontenc}
\usepackage[spanish]{babel} % Changes theorem->teorema, proof->demostración...

%%%%%%%%%%%%%%%%%%%%%%%%%%%%%%%%%%%%%%%%%%%%%%%
%%%%%%%%%%%%%%%% CONFIGURATION %%%%%%%%%%%%%%%%
%%%%%%%%%%%%%%%%%%%%%%%%%%%%%%%%%%%%%%%%%%%%%%%

\usetheme{default}
\setbeamertemplate{navigation symbols}{} % Removes the navigation symbols :_)

% Opens the PDF document in full screen by default :)
%\hypersetup{pdfpagemode=FullScreen}

%Language
\selectlanguage{spanish}

%%%%%%%%%%%%%%%%%%%%%%%%%%%%%%%%%%%%%%%%%%%%%%%
%%%%%%%%%%%%%%%%%%%% TITLE %%%%%%%%%%%%%%%%%%%%
%%%%%%%%%%%%%%%%%%%%%%%%%%%%%%%%%%%%%%%%%%%%%%%

\title[Teoría de Códigos]{Teoría de códigos y \\ polinomios torcidos}
% \subtitle[Breve introducción]{Una introducción a la teoría de códigos algebraica}
\author[A.]{Adrián}
\institute[UGR]{Universidad de Granada}
\date{\today}

% Frame whenever a new section begins
\AtBeginSection[]{
    \begin{frame}[plain]
        \begin{center}
                \huge{\insertsection}
        \end{center}
    \end{frame}
}

% Frame whenever a new subsection begins
\AtBeginSubsection[]{
    \begin{frame}[plain]
        \begin{center}
                \huge{\insertsubsection}
        \end{center}
    \end{frame}
}


%%%%%%%%%%%%%%%%%%%%%%%%%%%%%%%%%%%%%%%%%%%%%%%
%%%%%%%%%%%%%%%%%%% COLORS %%%%%%%%%%%%%%%%%%%%
%%%%%%%%%%%%%%%%%%%%%%%%%%%%%%%%%%%%%%%%%%%%%%%

%%%%%%%%%%%%%%%% DECLARATIONS %%%%%%%%%%%%%%%%%

\definecolor{c_gray_l}{HTML}{FAFAFA} % background light gray
\definecolor{c_metal}{HTML}{37474F}  % background dark metal blue

%%%%%%%%%%%%%%%% DEFINITIONS %%%%%%%%%%%%%%%%%*

\setbeamercolor*{default}{
    bg=c_metal,
    fg=c_gray_l
}

\setbeamercolor*{palette primary}{
    parent=default
}

\setbeamercolor*{palette secondary}{
    bg=c_gray_l,
    fg=c_metal
}

\setbeamercolor*{background canvas}{
    parent = palette primary
}

\setbeamercolor*{normal text}{
    fg = c_gray_l
}

\setbeamercolor*{titlelike}{
    parent = palette secondary
}

\setbeamercolor*{section in toc}{
    parent = palette primary
}

\setbeamercolor*{block title alerted}{
    bg = red!60,
    fg = c_gray_l
}

\setbeamercolor*{block body alerted}{
    parent = palette secondary
}

\setbeamercolor*{block body alerted}{
    parent = palette secondary
}

\setbeamercolor{structure}{fg=red!60}

\renewcommand{\a}{\mathbf{a}}
\newcommand{\x}{\mathbf{x}}
\newcommand{\y}{\mathbf{y}}
\newcommand{\z}{\mathbf{z}}
\newcommand{\e}{\mathbf{e}}
\renewcommand{\c}{\mathbf{c}}
\newcommand{\Fq}{\mathbb{F}_q}
\newcommand{\Fqn}{\mathbb{F}_q^n}
\newcommand{\Fqk}{\mathbb{F}_q^k}
\newcommand{\Fbm}{\mathbb{F}_{2^m}}
\newcommand{\Fqx}{\Fq[x]}
\newcommand{\Fx}{\Fq[x]/(x^n - 1)}
\newcommand{\Co}{C_{\theta}}
\newcommand{\Fqxo}{\Fq[x, \theta]}
\newcommand{\Fxx}{\Fq[x, \theta]/(x^n - 1)}
\newcommand{\tc}{$\theta$-cíclico\ }
\newcommand{\tcs}{$\theta$-cíclicos\ }

\theoremstyle{definition} % titulo negrita, cuerpo en recta
\newtheorem{teorema}{Teorema}[section]
\newtheorem{proposicion}{Proposicion}[section]
\newtheorem{lema}{Lema}[section]
\newtheorem{ejemplo}{Ejemplo}[section]
\newtheorem{definicion}{Definición}[section]
\newtheorem{nota}{Nota}[section]

\DeclareTextFontCommand{\emph}{\color{red!60}}

%%%%%%%%%%%%%%%%%%%%%%%%%%%%%%%%%%%%%%%%%%%%%%%
%%%%%%%%%%%%%%%%%% DOCUMENT %%%%%%%%%%%%%%%%%%%
%%%%%%%%%%%%%%%%%%%%%%%%%%%%%%%%%%%%%%%%%%%%%%%

\begin{document}

    \titlepage

    \begin{frame}[t]{Contenidos}
        \tableofcontents
    \end{frame}
    %--- Next Frame ---%

    % Ejemplo
    % \section{Ejemplo sección}
    % \begin{frame}{Ejemplo frame}{Subtitulo frame}
    %     \alert{Una alerta}
    %
    %     \begin{block}{Título de bloque}
    %         Texto en bloque
    %     \end{block}
    %
    %     Ejemplo frase
    %
    %     \begin{alertblock}{Ejemplo de bloque oculto 2 pasos}<2->
    %         Ejemplo de cuerpo de bloque
    %     \end{alertblock}
    %
    %
    % \end{frame}
    % %--- Next Frame ---%
    % \begin{frame} % si no pongo titulo y subtitulo no aparece
    %     \begin{alertblock}{Titulo de bloque}
    %         $ 2 + 2 = 4$
    %
    %         % para ocultar algo
    %         \uncover<2->{
    %             Estoy oculto!
    %         }
    %     \end{alertblock}
    % \end{frame}


    \section{Introducción}
    \begin{frame}{Los códigos correctores de errores}
        Los códigos correctores de errores se utilizan para detectar y corregir errores que ocurren cuando la información se transmite por una canal con ruido.

        \ \\

        Algunas aplicaciones;

        \begin{itemize}
            \item Reproductores de CD
            \item Envío de fotografías del espacio a la Tierra
        \end{itemize}

        \ \\

        Para ello añaden redundancia al mensaje.
    \end{frame}

    \begin{frame}{El código de repetición}
        Caso 1.
        \begin{itemize}
            \item 1 $\rightarrow$ <<sí>>
            \item 0 $\rightarrow$ <<no>>
            \item ¿Qué pasa si hay un error?
        \end{itemize}

        \ \\

        Caso 2.
        \begin{itemize}
            \item 111 $\rightarrow$ <<sí>>
            \item 000 $\rightarrow$ <<no>>
            \item Se puede corregir un error:
            \begin{itemize}
                \item {001, 010, 100} $\rightarrow$ 000
                \item {110, 101, 011} $\rightarrow$ 111
            \end{itemize}
        \end{itemize}
    \end{frame}

    \section{Preliminares}

    \begin{frame}{Primeras definiciones}
        % \begin{alertblock}{Alfabeto}
        %     Conjunto de símbolos que utiliza un código. En nuestro caso, utilizaremos los cuerpos finitos $\Fq$.
        % \end{alertblock}
        Llamaremos \emph{alfabeto} al conjunto de símbolos que utiliza un código. En nuestro caso, utilizaremos los cuerpos finitos $\Fq$.

        \ \\

        Un \emph{código de bloque} consiste en una función codificadora $E: \mathbb{F}_q^k \to \mathbb{F}_q^n$ y una función decodificadora $D: \mathbb{F}_q^n \to \mathbb{F}_q^k $, con $k < n$. Los elementos de $\mathbb{F}_q^k$ se llaman \emph{mensajes}, los de $\Fqn$ palabras y los de $Im(E) \subset \Fqn$ se llaman \emph{palabras código}.
    \end{frame}

    \begin{frame}{Primeras definiciones}
        % Mecionar: tenemos el mensaje 1 y el mensaje 0.
        % La codificación lleva mensajes en palabras de ćodigo
        Código de repetición:
        \begin{itemize}
            \item 111 $\rightarrow$ <<sí>> $\equiv$ 1
            \item 000 $\rightarrow$ <<no>> $\equiv$ 0
        \end{itemize}

        \ \\

        111 y 000 son las palabras código. La codificación y la decodificación son las siguientes:
        \begin{align*}
            E(1) &= 111 \\
            E(0) &= 000 \\
            D(111) &= 1 = D(110) = D(101) = D(011) \\
            D(000) &= 0 = D(001) = D(010) = D(100)
        \end{align*}
    \end{frame}

    \begin{frame}{Paramétros importantes}
        \begin{itemize}
            \item La longitud de código, \emph{n}.
            \item El número total de palabras código, \emph{M}.
            \item La distancia mínima entre pares de palabras código, \emph{d}.
        \end{itemize}

        La distancia (de Hamming) entre dos palabra código $d(\x, \y)$ es el número de posiciones en los que $\x$ e $\y$ difieren. Esta distancia es una métrica en $\Fqn$.

        \ \\

        Código de repetición: \uncover<2->{$n = 3, M = 2, d = 3$}
        \begin{itemize}
            \item 111 $\rightarrow$ <<sí>>
            \item 000 $\rightarrow$ <<no>>
        \end{itemize}

    \end{frame}

    \begin{frame}{Corrigiendo y detectando errores}
        \begin{itemize}
            \item El \emph{decodificador} debe decidir que palabra código fue transmitida.
            \item Esquema básico de decodificación \emph{vecino más cercano}: elegir la palabra código más <<cercana>> a la palabra recibida.
            \item Supone que el canal es \emph{simétrico}: \begin{itemize}
                \item Cada símbolo transmitido tiene la misma probabilidad de recibirse erróneamente.
                \item Si un símbolo se recibe erróneamente, cada uno de los otros posibles errores es equiprobable.
            \end{itemize}
        \end{itemize}
    \end{frame}

    \begin{frame}{Corrigiendo y detectando errores}
        \begin{alertblock}{Proposición}
            Un código con una distancia mínima $d$ puede detectar hasta $d - 1$ errores y corregir hasta $\lfloor (d - 1)/2 \rfloor$ errores.
        \end{alertblock}

        \ \\

        Código de repetición: $n = 3, M = 2, d = 3$
        \begin{itemize}
            \item 111 $\rightarrow$ <<sí>>
            \item 000 $\rightarrow$ <<no>>
        \end{itemize}
        \uncover<2->{Detecta hasta 2 errores y corrige hasta 1 error.}

        % Interpretación geométrica en clase
    \end{frame}

    \section{Teoría de códigos algebraica}
    \begin{frame}{Códigos lineales}
        Un \emph{código lineal}  de longitud $n$ sobre $\Fq$ es un subespacio vectorial del espacio vectorial $\Fqn$.

        \ \\
        % porqué codigos lineales

        Un $[n, k]$-código lineal es un código lineal de longitud $n$ sobre sobre $\Fq$ con dimensión $k$ como subespacio vectorial.

        \ \\

        Podemos identificar el espacio de mensajes con el espacio vectorial $\mathbb{F}_q^k$. Así, las funciones de codificación/decodificación serían:
        $$
            E: \mathbb{F}_q^k \to \mathbb{F}_q^n, \
            D: \mathbb{F}_q^n \to \mathbb{F}_q^k
        $$
        % modificamos mensajes de longitud $k$ para obtener palabras código de longitud $n$, con $n \geq k$ y viceversa
    \end{frame}

    \begin{frame}{Códigos lineales}
        \begin{alertblock}{Proposición}
            Sea $C$ un código lineal. Toda combinación lineal de palabras código de $C$ es una palabra código de $C$.
        \end{alertblock}

        \begin{alertblock}{Proposición}
            La distancia mínima $d(C)$ de un código lineal $C$ es igual a la mínima distancia entre los pares $(\x, 0)$ con $\x$ no nulo.
        \end{alertblock}
        % ayuda a encontrar la distancia mínima de un código lineal
    \end{frame}

    \begin{frame}{Códigos lineales}
        \begin{alertblock}{Límite de Singleton}
            Si $C$ es un $[n, k]$-código lineal entonces $d \leq n - k + 1$.
        \end{alertblock}

        \ \\

        A los códigos que consiguen la igualdad se les llama \emph{códigos separables de distancia máxima}.

        \ \\

        Ejemplo: \emph{Reed-Solomon}. Utilizado en CDs, DVDs, Blu-ray, QR, \dots.
    \end{frame}

    \begin{frame}{Códigos lineales}{Matriz generatriz y codificación}
        Una matriz G de tamaño $k \times n$ cuyas filas forman una base para un $[n, k]$-código lineal se llama \emph{matriz generatriz} del código.

        \ \\

        Si $C$ es un $[n, k]$-código lineal con matriz generatriz $G$, entonces la función de codificación puede escribirse como:
       \begin{align*}
           E: \Fqk &\to \Fqn\\
           \x &\mapsto \x G
       \end{align*}
    \end{frame}

    \begin{frame}{Códigos lineales}{Matriz generatriz y codificación. Ejemplo}
        Sea $C$ el código linear sobre $\mathbb{F}_2^4$ de dimensión 2 con matriz generatriz
        $$
        G = \begin{pmatrix}
            1 & 0 & 1 & 1 \\
            0 & 1 & 0 & 1
        \end{pmatrix}
        $$

        \ \\

        Multiplicando los elementos $\mathbb{F}_2^2$ con $G$ obtenemos el código $C$:
        $$
            \left\{
            \uncover<2->{0000,}
            \uncover<3->{1011,}
            \uncover<4->{0101,}
            \uncover<5->{1100}
            \right\}
        $$
    \end{frame}


    \begin{frame}{Códigos lineales}{Clases laterales y decodificación}
        Sea $C$ un $[n, k]$-código lineal sobre $\Fq$ y $\a$ un vector de $\Fqn$. El conjunto $\a + C = \{\a + \x\ |\ \x \in C\}$ se llama una \emph{clase lateral} de $C$.

        \ \\

        \begin{alertblock}{Proposición}
            Si $C$ es un $[n, k]$ código lineal sobre $\Fq$, entonces cada vector de $\Fqn$ está en alguna clase lateral de $C$, cada clase lateral contiene $q^k$ vectores y dos clases laterales son idénticas o disjuntas.
        \end{alertblock}
    \end{frame}


    \begin{frame}{Códigos lineales}{Clases laterales y decodificación}
        Esquema \emph{vecino más cercano}:
        \begin{enumerate}
            \uncover<2->{\item Particionamos $\Fqn$ en clases laterales de $C$.}
            \uncover<3->{\item Para cada clase lateral, elegimos un representante con número de componentes no nulas mínimo.}
        \end{enumerate}
        \uncover<4->{Al recibir una palabra $\y$:}
        \begin{enumerate}
            \uncover<5->{\item Buscamos la clase lateral que contiene a $\y$. Esta será de la forma $\e + C$}
            \uncover<6->{\item $\y - \e$ fue la palabra de código enviada.}
        \end{enumerate}
    \end{frame}

    \begin{frame}{Códigos lineales}{Clases laterales y decodificación}
        Sea $C = \left\{0000, 1011, 0101, 1100\right\}$.

        \ \\

        Las clases laterales de $C$ son:
        \begin{align*}
            \uncover<2->{0000 + C &= \left\{0000, 1011, 0101, 1110 \right\}} \\
            \uncover<3->{1000 + C &= \left\{1000, 0011, 1101, 0110 \right\}} \\
            \uncover<4->{0100 + C &= \left\{0100, 1111, 0001, 1010 \right\}} \\
            \uncover<5->{0010 + C &= \left\{0010, 1001, 0111, 1100 \right\}}
        \end{align*}

    \end{frame}


    \begin{frame}{Códigos cíclicos}
        Un código lineal $C$ de longitud $n$ es \emph{cíclico} si verifica:
        $$
            (c_0, c_1, \dots, c_{n-1}) \in C \implies (c_{n-1}, c_0, \dots, c_{n-1}) \in C
        $$

        \ \\

        Los códigos cíclicos se pueden implementar de forma eficiente usando dispositivos hardware llamados registros de desplazamiento.
    \end{frame}

    \begin{frame}{Códigos cíclicos}{Representación polinomial}
        \begin{align*}
            \phi: \Fqn &\to \Fx \\
            (c_0, c_1, \dots, c_{n-1}) & \mapsto c_0 + c_1 x + \cdots + c_{n-1} x^{n-1}
        \end{align*}

        \begin{itemize}
            \item $\phi$ es un isomorfismo de espacios vectoriales.
        \end{itemize}

        Son equivalentes:
        \begin{align*}
            i)&\ (c_0, c_1, \dots, c_{n-1}) \in C \implies (c_{n-1}, c_0, \ldots, c_{n-2}) \in C \\
            ii)&\ x (c_0 + c_1 x + \cdots + c_{n-1} x^{n-1}) \in \phi(C)
        \end{align*}
    \end{frame}

    \begin{frame}{Códigos cíclicos}{Representación polinomial}
        La equivalencia se basta en la siguiente igualdad en $\Fx$:
        \begin{align*}
            x &(c_0 + c_1 x + \cdots + c_{n-1} x^{n-1}) \\
                \uncover<2->{&= x c_0 + x c_1 x + \cdots + x c_{n-1} x^{n-1} \\}
                \uncover<3->{&= c_0 x + c_1 x^2  + \cdots + c_{n-1} x^{n} \\}
                \uncover<4->{&= c_{n-1} + c_0 x + c_1 x^2 + \cdots + c_{n-2} x^{n-1}}
        \end{align*}
    \end{frame}

    \begin{frame}{Códigos cíclicos}{Representación polinomial}
        Si $f(x) \in \phi(C)$, entonces $$\left\{x f(x), x^2 f(x), x^3 f(x), \dots, x^{n-1} f(x)\right\}$$ también están en $\phi(C)$.

        \ \\

        Usando la linealidad de $C$ se deduce: $$\forall p(x) \in \Fx,\ p(x) f(x) \in \phi(C)$$
    \end{frame}

    \begin{frame}{Códigos cíclicos}{Representación polinomial}
        \begin{alertblock}{Teorema}
            Los códigos cíclicos de longitud $n$ sobre $\Fq$ son justamente los ideales del anillo $\Fx$
        \end{alertblock}

        \begin{alertblock}{Proposición}
            Sea $C$ un código cíclico de longitud $n$. Entonces $C$ está generado por un divisor  $g(x)$ de $(x^n - 1)$ en $\Fqx$. Si $g(x)$ mónico y de grado mínimo, $g(x)$ es único y se llama el \emph{polinomio generador} de $C$.
        \end{alertblock}
    \end{frame}

    \begin{frame}{Códigos cíclicos}{Codificación}
        Dado un mensaje $\a = (a_0, \dots, a_{n-1})$, y su representación polinomial $a(x)$, mediante multiplicación de polinomios podemos realizar la \emph{codificación}:
        \begin{align*}
            E: \Fx &\to \Fx \\
            a(x) &\mapsto a(x)g(x)
        \end{align*}
    \end{frame}


    \begin{frame}{Códigos cíclicos}{Matriz generatriz}
        \begin{lema}
            Sea $C$ un código cíclico con polinomio generador $g(x) = g_0 + g_1 x + \cdots + g_r x^r$ de grado $r$. Entonces la dimensión de $C$ es $n - r$ y una matriz generatriz para $C$ es la siguiente matriz $(n - r) \times n$:
            $$
            G = \begin{pmatrix}
                    g_0 & g_1 & \cdots & g_r & 0 & 0 & \cdots & 0 \\
                    0 & g_0 & g_1 & \cdots & g_r & 0 & \cdots & 0 \\
                    0 & 0 & g_0 & g_1 & \cdots & g_r & \ddots & \vdots \\
                    \vdots & \vdots & \ddots & \ddots & \ddots & \cdots & \ddots & 0  \\
                    0 & 0 & \cdots & 0 & g_0 & g_1 & \cdots & g_r
                \end{pmatrix}
            $$
        \end{lema}
    \end{frame}

    \begin{frame}{Códigos cíclicos}{Matriz generatriz. Demostración}
        $g_0$ no puede ser 0 (en caso contrario $(0, g_1, \ldots, g_r, 0, \ldots) \in C \Rightarrow (g_1, \ldots, g_r, 0, \ldots ) \in C$)

        \ \\

        Las $n - r$ filas de la matriz $G$ son linealmente independientes por la forma escalonada de la matriz.

        \ \\

        Estas $n - r$ filas representan los polinomios código $g(x), x g(x), \ldots, x^{n - r - 1}g(x)$. Veamos que forman un sistema de generadores.

    \end{frame}
    \begin{frame}{Códigos cíclicos}{Matriz generatriz. Demostración}
        $c(x) \in C \implies \exists m(x) \in \Fx$ tal que:
        \begin{align*}
            c(x) &= m(x) g(x) \\
                 &= (m_0 + m_1 x + \cdots + m_{n - r - 1} x^{n - r - 1}) g(x) \\
                 &= m_0 g(x) + m_1 x g(x) + \cdots m_{n - r - 1} x^{n - r - 1} g(x)
        \end{align*}
        $\implies$ G es una matriz generatriz de $C$ y la dimensión de $C$ es $n - r \qed$.
    \end{frame}

    \begin{frame}{Códigos cíclicos}{Listado de códigos}
        \begin{alertblock}{Proposición}
            Hay una correspondencia uno a uno entre los divisores mónicos de $x^n - 1$ en $\Fqx$ y los códigos cíclicos sobre $\Fq$ de longitud $n$.
        \end{alertblock}

        Este resultado facilita el listado  de todos los códigos cíclicos de una longitud dada. Veamos un ejemplo.
    \end{frame}

    \begin{frame}{Códigos cíclicos}{Listado de codigos}
        Vamos a calcular todos los códigos cíclicos sobre $\mathbb{F}_2$ de longitud 4. La factorización de $x^4 + 1$ es $(x + 1)^4$, luego sus factores mónicos son:
        \begin{align*}
            \uncover<2->{g_0(x) &= 1} \\
            \uncover<3->{g_1(x) &= x + 1} \\
            \uncover<4->{g_2(x) &= (x + 1)^2 = x^2 + 1} \\
            \uncover<5->{g_3(x) &= (x + 1)^3 = x^3 + x^2 + x + 1} \\
            \uncover<6->{g_4(x) &= (x + 1)^4 = x^4 + 1} \\
        \end{align*}
    \end{frame}

    \begin{frame}{Códigos cíclicos}{Listado de códigos. Ejemplo}
        El primer y el último factor generan los códigos triviales $\mathbb{F}_2[x]/(x^4 + 1)$ y $\{0\}$ respectivamente. El resto de polinomios generan los códigos:
        \begin{align*}
            \uncover<2->{g_1(x) &\rightarrow C_1 = \left\{0, x + 1, x^{2} + 1, x^{2} + x, x^{3} + 1, \right. \\
            & \left. x^{3} + x, x^{3} + x^{2}, x^{3} + x^{2} + x + 1\right\}} \\
            \uncover<3->{g_2(x) &\rightarrow C_2 = \left\{0, x^{2} + 1, x^{3} + x, x^{3} + x^{2} + x + 1\right\}} \\
            \uncover<4->{g_3(x) &\rightarrow C_3 = \left\{0, x^{3} + x^{2} + x + 1\right\}}
        \end{align*}

        \uncover<5->{$C_1$ es un código $[4, 3]$-cíclico, $C_2$ es un código $[4, 2]$-cíclico y $C_3$ es un código $[4, 1]$-cíclico.}
    \end{frame}

    \section{Códigos cíclicos torcidos}

    \begin{frame}{Códigos cíclicos torcidos}
        Sea $\theta$ un automorfismo de $\Fq$. Un código \emph{$\theta$-cíclico} es un código lineal $\Co$ verificando:
        $$
        (a_0, a_1, \dots, a_{n-1}) \in \Co \Rightarrow (\theta(a_{n-1}), \theta(a_0), \dots, \theta(a_{n-2}))
        $$

        \ \\

        Consideramos el \emph{anillo de polinomios torcidos} sobre $\Fq$, esto es, la extensión de Ore:
        $$
            \Fqxo = \{a_0 + a_1 x + \cdots + a_{n-1} x^{n-1},\ a_i \in \Fq,\ n \in \mathbb{N}\}
        $$
        donde la regla de multiplicación es $x a = \theta(a) x$.
    \end{frame}


    \begin{frame}{Códigos cíclicos torcidos}{Anillo de polinomios torcidos}
        $\Fqxo$
        \begin{itemize}
            \item tiene un algoritmo de división euclídeo por la izquierda y por la derecha.
            \item es un dominio de ideales principales por la izquierda y por la derecha.
        \end{itemize}

        \ \\

        Denotamos por $K \subset \Fq$ al subcuerpo cuyos elementos se quedan fijos por $\theta$.
    \end{frame}

    \begin{frame}{Códigos cíclicos torcidos}{Centro del anillo de polinomios torcidos}
        \begin{alertblock}{Proposición}
            Los elementos centrales de $\Fqxo$ son de la forma
            $\sum_{i = 0}^{m}c_i x^{i |\theta|}$ donde $c_i \in K$ y $|\theta|$ es el orden del automorfismo $\theta$. Simbólicamente:
            $$
                Z(\Fqxo) = K[x^{|\theta|}]
            $$
        \end{alertblock}
    \end{frame}

    \begin{frame}{Códigos cíclicos torcidos}{Centro del anillo de polinomios torcidos. Demostración}
        Sea $g(x) = \sum_{i = 0}^{m}g_i x^i \in Z(\Fqxo)$.
        \begin{align*}
            g(x) a_0 &= \sum_{i = 0}^{m}g_i \theta^i(a_0) x^i \\
            a_0 g(x) &= \sum_{i = 0}^{m}a_0 g_i x^i
        \end{align*}
        $\implies a_0 = \theta^i(a_0)\ \forall a_0 \in \Fq \implies g(x) \in \Fq[x^{|\theta|}]$
    \end{frame}

    \begin{frame}{Códigos cíclicos torcidos}{Centro del anillo de polinomios torcidos. Demostración}
        $g(x) = \sum_{i = 0}^{m}g_i x^i$
        \begin{align*}
            g(x)\  b_1 x &= \sum_{i = 0}^{m}g_{i} \theta^{i}(b_1) x^{i+1} \\
            b_1 x \  g(x) &= \sum_{i = 0}^{m}b_1 \theta(g_{i}) x^{i+1}
        \end{align*}
        $\implies g_{i} = \theta(g_{i}) \implies g_i \in K \implies g(x) \in K[x^{|\theta|}]$
    \end{frame}

    \begin{frame}{Códigos cíclicos torcidos}{Centro del anillo de polinomios torcidos. Demostración}
        Sea $c\ x^{j|\theta|} \in K[x^{|\theta|}]$,
        $\ f(x) = \sum_{i = 0}^{m}f_i x^i \in \Fqxo$.

        \begin{align*}
            c\ x^{j|\theta|} \times f(x) &=
                \sum_{i = 0}^{m}c\ \theta^{j|\theta|}(f_{i}) x^{i+j|\theta|} \\
            f(x) \times c\ x^{j|\theta|} &=
                \sum_{i = 0}^{m}f_{i} \theta^{i}(c) x^{i+j|\theta|}
        \end{align*}
        $\implies c\ x^{j|\theta|} \in Z(\Fqxo)$
        % \implies K[x^{|\theta|}] \subset Z(\Fqxo)$ (por linealidad).

        \ \\

        Por linealidad, $K[x^{|\theta|}] \subset Z(\Fqxo) \qed$
    \end{frame}

    \begin{frame}{Códigos cíclicos torcidos}{Resultados conocidos}
        \begin{alertblock}{Proposición}
            Los elementos centrales de $\Fqxo$ son los generadores de los ideales biláteros de $\Fqxo$ y $(x^n - 1) \subset \Fqxo$ es un ideal bilátero.
        \end{alertblock}
        \begin{alertblock}{Proposición}
            El anillo $\Fxx$ es un anillo de ideales principales por la izquierda cuyos ideales por la izquierda están generados por un divisor de $x^n - 1$ en $\Fqxo$.
        \end{alertblock}
    \end{frame}

    \begin{frame}{Códigos cíclicos torcidos}{Representación polinomial <<torcida>>}
        \begin{align*}
            \phi: \Fqn &\to \Fxx \\
            (c_0, c_1, \dots, c_{n-1}) & \mapsto c_0 + c_1 x + \cdots + c_{n-1} x^{n-1}
        \end{align*}

        \begin{itemize}
            \item $\phi$ es un isomorfismo de $\Fq$-módulos
        \end{itemize}

        Son equivalentes:
        \begin{align*}
            &\ (c_0, c_1, ..., c_{n-1}) \in C \Rightarrow (\theta(c_{n-1}), \theta(c_0), ..., \theta(c_{n-2})) \in C \\
            &\ x (c_0 + c_1 x + \ldots + c_{n-1} x^{n-1}) \in \phi(C)
        \end{align*}
    \end{frame}

    \begin{frame}{Códigos cíclicos torcidos}{Representación polinomial <<torcida>>}
        La equivalencia se basta en la siguiente igualdad en $\Fxx$:
        \begin{align*}
            x &(c_0 + c_1 x + \cdots + c_{n-1} x^{n-1}) \\
                \uncover<2->{&= x c_0 + x c_1 x + \cdots + x c_{n-1} x^{n-1}  \\}
                \uncover<3->{&= \theta(c_0) x + \theta(c_1) x^2  + \cdots + \theta(c_{n-1}) x^{n} \\}
                \uncover<4->{&= \theta(c_{n-1}) + \theta(c_0) x + \theta(c_1) x^2 + \cdots + \theta(c_{n-2}) x^{n-1}}
        \end{align*}
    \end{frame}

    \begin{frame}{Códigos cíclicos torcidos}{Representación polinomial <<torcida>>}
        Si $f(x) \in \phi(C)$, entonces $$\left\{x f(x), x^2 f(x), x^3 f(x), \dots, x^{n-1} f(x)\right\}$$ también están en $\phi(C)$.

        \ \\

        Usando la linealidad de $C$ se deduce: $$\forall p(x) \in \Fxx,\ p(x) f(x) \in \phi(C)$$
    \end{frame}


    \begin{frame}{Códigos cíclicos torcidos}{Representación polinomial <<torcida>>}
        \begin{alertblock}{Teorema}
            Los códigos \tcs de longitud $n$ sobre $\Fq$ son justamente los ideales por la izquierda del anillo $\Fxx$.
        \end{alertblock}
    \end{frame}

    \begin{frame}{Códigos cíclicos torcidos}{Representación polinomial <<torcida>>}
        Un factor por la derecha de grado $n - k$ de $x^n - 1$ genera un $[n, k]$-código lineal.

        \ \\

        $\Fqxo$ no es en general un dominio de factorización única. Como hay muchos mas factores por la derecha que en el caso conmutativo, existen más codigos \tcs que cíclicos.

        \ \\

        Nótese que aunque la factorización no sea única, los grados de los polinomios torcidos irreducibles en la factorización de un elemento en $\Fqxo$ si son únicos salvo permutación.

    \end{frame}

    \begin{frame}{Códigos cíclicos torcidos}{Ejemplo}
        Vamos a buscar todos los [4, 2]-códigos \tcs con $\theta(a) = a^2$ sobre $\mathbb{F}_4$.

        \ \\

        Sea $\alpha$ un generador del grupo multiplicativo de $\mathbb{F}_4$, esto es, un cero de $z^2 + z + 1 \in \mathbb{F}_2$ en $\overline{\mathbb{F}_2}$. Así, $\mathbb{F}_4 = \left\{0, 1, \alpha, \alpha^2 \right\}$.

        \ \\

        Calculamos los factores mónicos por la derecha de grado dos de $x^4 + 1 \in \mathbb{F}_4[x, \theta]$.

    \end{frame}

    \begin{frame}{Códigos cíclicos torcidos}{Ejemplo}
        \begin{align*}
            g_1(x) &= x^2 + 1 \\
            g_2(x) &= x^2 + \alpha x + \alpha ^2 \\
            g_3(x) &=  x^2 + \alpha^2 x + \alpha \\
            g_4(x) &=  x^2 + \alpha^2 x + \alpha^2 \\
            g_5(x) &=  x^2 + x + \alpha \\
            g_6(x) &=  x^2 + x + \alpha^2 \\
            g_7(x) &= x^2 + \alpha x + \alpha
        \end{align*}

    \end{frame}
    \begin{frame}{Códigos cíclicos torcidos}{Ejemplo}

        Las factorizaciones correspondientes son:
        \begin{align*}
            x^4 + 1 &= (x^2 + 1)(x^2 + 1) \\
                    &= (x^2 + \alpha x + \alpha)(x^2 + \alpha x + \alpha ^2)\\
                    &= (x^2 + \alpha^2 x + \alpha^2)(x^2 + \alpha^2 x + \alpha)\\
                    &= (x^2 + \alpha^2 x + \alpha)(x^2 + \alpha^2 x + \alpha^2)\\
                    &= (x^2 + x + \alpha^2)(x^2 + x + \alpha)\\
                    &= (x^2 + x + \alpha)(x^2 + x + \alpha^2)\\
                    &= (x^2 + \alpha x + \alpha ^2)(x^2 + \alpha x + \alpha)
        \end{align*}

    \end{frame}
    \begin{frame}{Códigos cíclicos torcidos}{Ejemplo}

        Como  $x^4 + 1 = (x + 1)(x + 1)(x + 1)(x + 1)$, los factores irreducibles de $x^4 + 1 \in \mathbb{F}_4[x, \theta]$ en cualquier descomposicón son todos de grado uno. Por lo tanto, ninguno de los $g_i(x)$ anteriores es irreducible.

    \ \\

        El polinomio $g_1$ genera el único [4, 2]-código cíclico sobre $\mathbb{F}_4$. Los otros polinomios generan un código $[4, 2]$ (los seis códigos son equivalentes).
    \end{frame}

    \begin{frame}[plain]
        Para más información:

        \center{\emph{\url{www.github.com/ranea/teoriacodigos}}}

    \end{frame}

    \begin{frame}[plain]
        %\normalsize{}
        \huge{GRACIAS}
        \vspace{2mm}
        \hrule
    \end{frame}

\end{document}
